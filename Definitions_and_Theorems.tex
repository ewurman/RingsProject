\documentclass{article}

\usepackage{fancyhdr}
\usepackage{extramarks}
\usepackage{amsmath}
\usepackage{amsthm}
\usepackage{amsfonts}
\usepackage{tikz}
\usepackage[plain]{algorithm}
\usepackage{algpseudocode}

\usetikzlibrary{automata,positioning}
%
% Basic Document Settings
%

\topmargin=-0.45in
\evensidemargin=0in
\oddsidemargin=0in
\textwidth=6.5in
\textheight=9.0in
\headsep=0.25in

\linespread{1.1}

\pagestyle{fancy}
\lhead{\hmwkAuthorName}
\chead{\ClassName: \hmwkTitle}
\rhead{\hmwkDueDate}


%
% Homework Details
%   - Title
%   - Due date
%   - Class
%   - Author
%

\newcommand{\hmwkTitle}{Final Project Definitions and Theorems}
\newcommand{\hmwkDueDate}{Spring 2018}
\newcommand{\ClassName}{Math 3702}
\newcommand{\hmwkAuthorName}{\textbf{Erik, Bo, Phillip}}


%
%Helper functions
%
\newcommand{\separate}{\bigskip}
\newcommand{\smallseparate}{\medskip}
\newcommand{\separateprobs}{\vspace{.25in}}


%Math Functions:
\newcommand{\C}{\mathbb{C}}
\newcommand{\Z}{\mathbb{Z}}
\newcommand{\Q}{\mathbb{Q}}
\newcommand{\R}{\mathbb{R}}
\newcommand{\N}{\mathbb{N}}
\newcommand{\I}{\mathbb{I}}
\newcommand{\M}{\mathbb{M}}

\setlength\parindent{0pt}

\begin{document}

Definition: \textbf{Linear Code}\\ An $(n,k)$ \textit{linear code} over a finite field $F$ is a $k$-dimensional subspace $V$ of the vector space \[F^n = \underbrace{F\oplus F\oplus \dots \oplus F}_{\text{$n$ copies}}\]
over $F$. The members of $V$ are called the \textit{code words}. The ratio $k/n$ is called the \textit{information rate} of the code. When $F$ is $\Z_2$, the code is called binary.

Definition: \textbf{Hamming Distance, Hamming Weight}\\The \textit{Hamming distance} between two vectors of a vector space is the number of components in which they differ. The \textit{Hamming weight} of a vector is the umber of nonzero components of the vector.\\ We will use $d(u,v)$ to denote the Hamming distance between the vectors $u$ and $v$ and wt$(u)$ for the Hamming weight of the vector $u$.

\textit{\textbf{Theorem} Properties of Hamming Distance and Hamming Weight}\\
For any vectors $u,v$ and $w$ of a linear code, $d(u,v)\leq d(u,w)+d(w,v)$ and $d(u,v)=wt(u-v)$.

\textit{\textbf{Theorem} Correcting Capability of a Linear Code}\\
If the Hamming weight of every nonzero code word in a linear code is at least $2t+1$, then the code can correct for any $t$ or fewer errors. Furthermore, the same code can detect any $2t$ or fewer errors..


\end{document}
