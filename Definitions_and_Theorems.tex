\documentclass{article}

\usepackage{fancyhdr}
\usepackage{extramarks}
\usepackage{amsmath}
\usepackage{amsthm}
\usepackage{amsfonts}
\usepackage{tikz}
\usepackage[plain]{algorithm}
\usepackage{algpseudocode}
\usepackage[labelsep=none]{caption}

\usetikzlibrary{automata,positioning}
%
% Basic Document Settings
%

\topmargin=-0.45in
\evensidemargin=0in
\oddsidemargin=0in
\textwidth=6.5in
\textheight=9.0in
\headsep=0.25in

\linespread{1.1}

\pagestyle{fancy}
\lhead{\hmwkAuthorName}
\chead{\ClassName: \hmwkTitle}
\rhead{\hmwkDueDate}


%
% Homework Details
%   - Title
%   - Due date
%   - Class
%   - Author
%

\newcommand{\hmwkTitle}{Final Project \textit{Definitions and Theorems}}
\newcommand{\hmwkDueDate}{Spring 2018}
\newcommand{\ClassName}{Math 3702}
\newcommand{\hmwkAuthorName}{\textbf{Erik, Bo, Phillip}}


%
%Helper functions
%
\newcommand{\separate}{\bigskip}
\newcommand{\smallseparate}{\medskip}
\newcommand{\separateprobs}{\vspace{.25in}}


%Math Functions:
\newcommand{\C}{\mathbb{C}}
\newcommand{\Z}{\mathbb{Z}}
\newcommand{\Q}{\mathbb{Q}}
\newcommand{\R}{\mathbb{R}}
\newcommand{\N}{\mathbb{N}}
\newcommand{\I}{\mathbb{I}}
\newcommand{\M}{\mathbb{M}}

\setlength\parindent{0pt}

\newtheorem{theorem}{Theorem}
\newtheorem{corollary}{Corollary}
\newtheorem{lemma}{Lemma}
\theoremstyle{definition}
\newtheorem{definition}{Definition}
\newtheorem{example}{Example}

\begin{document}
\begin{definition}{Linear Code}\\An $(n,k)$ \textit{linear code} over a finite field $F$ is a $k$-dimensional subspace $V$ of the vector space \[F^n = \underbrace{F\oplus F\oplus \dots \oplus F}_{\text{$n$ copies}}\]
over $F$. The members of $V$ are called the \textit{code words}. The ratio $k/n$ is called the \textit{information rate} of the code. When $F$ is $\Z_2$, the code is called binary.
\end{definition}

\begin{example}{The Hamming (7,4) Code.}\\
Assuming that our message consists of all possible 4-tuples of 0's and 1's (i.e., we wish to send a sequence of 0's and 1's of length 4). Encoding will be done by viewing these messages as four-dimensional vectors over the field $\Z_2$ and multiplying each of the 16 possible messages on the right by the matrix
\[G=\begin{bmatrix}
1&0&0&0&0&1&1\\
0&1&0&0&1&0&1\\
0&0&1&0&1&1&0\\
0&0&0&1&1&1&1
\end{bmatrix}\]
The resulting seven-dimensional vectors are called \textit{code words.} See Table~\ref{tab:hamming(7,4)}.
\begin{table}[ht!]
    \centering
    \begin{tabular}{c c c}
        \hline
        Message & Encoder $G$ & Code Word\\
        \hline
        0000 & $\rightarrow$ & 0000000 \\
        0001 & $\rightarrow$ & 0001111 \\
        0010 & $\rightarrow$ & 0010110 \\
        0100 & $\rightarrow$ & 0100101 \\
        1000 & $\rightarrow$ & 1000011 \\
        1100 & $\rightarrow$ & 1100110 \\
        1010 & $\rightarrow$ & 1010101 \\
        1001 & $\rightarrow$ & 1001100 \\
        0110 & $\rightarrow$ & 0110011 \\
        0101 & $\rightarrow$ & 0101010 \\
        0011 & $\rightarrow$ & 0011001 \\
        1110 & $\rightarrow$ & 1110000 \\
        1101 & $\rightarrow$ & 1101001 \\
        1011 & $\rightarrow$ & 1011010 \\
        0111 & $\rightarrow$ & 0111100 \\
        1111 & $\rightarrow$ & 1111111 \\
    \end{tabular}
    \caption{}
    \label{tab:hamming(7,4)}
\end{table}
\end{example}

\begin{definition}{Hamming Distance, Hamming Weight}\\The \textit{Hamming distance} between two vectors of a vector space is the number of components in which they differ. The \textit{Hamming weight} of a vector is the umber of nonzero components of the vector.\\ We will use $d(u,v)$ to denote the Hamming distance between the vectors $u$ and $v$ and wt$(u)$ for the Hamming weight of the vector $u$.
\end{definition}

\begin{theorem}{Properties of Hamming Distance and Hamming Weight}\\
For any vectors $u,v$ and $w$ of a linear code, $d(u,v)\leq d(u,w)+d(w,v)$ and $d(u,v)=wt(u-v)$.
\end{theorem}

\begin{theorem}{Correcting Capability of a Linear Code}\\
If the Hamming weight of every nonzero code word in a linear code is at least $2t+1$, then the code can correct for any $t$ or fewer errors. Furthermore, the same code can detect any $2t$ or fewer errors..
\end{theorem}


\end{document}
